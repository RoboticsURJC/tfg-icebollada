\cleardoublepage

\chapter*{Resumen\markboth{Resumen}{Resumen}}
Gracias al desarrollo tecnológico, la robótica está adquiriendo cada vez más importancia en diferentes campos, facilitando y simplificando el trabajo de las personas debido a la eficiencia y automatización de sistemas previamente manejados por humanos.\\

Uno de los campos claves en robótica es la inteligencia artificial, que pretende replicar el funcionamiento del cerebro humano. A su vez, uno de sus campos más importantes es la visión artificial, que pretende imitar el funcionamiento de la visión humana. Para ello, es importante el entrenamiento de algoritmos. Los más usados son algoritmos de Deep Learning, que tratan de replicar la estructura de las neuronas.\\

Para este TFG se ha creado un sistema de monitorización para el control y estudio del bienestar de ratones. Se ha contactado con los investigadores del Laboratorio de Bienestar e Investigación Animal de la UAH para el conocimiento de los requisitos reales, los cuales se han analizado y se les ha dado solución mediante el desarrollo de una interfaz de usuario que ofrece de forma simple los datos recopilados por el sistema.\\

Para la implementación de este sistema, se ha utilizado la placa Raspberry Pi 4, lo que permite que el sistema sea \textit{low-cost}. La interfaz se ha creado en Node-Red, donde se han recogido las lecturas de los distintos sensores así como los servidores web de las cámaras, creados con Flask. Además se ha creado un dataset con imágenes de ratones, que ha permitido entrenar un modelo de detección en YOLO, entrenado a través de una red neuronal convolucional, un tipo de algoritmo de Deep Learning.\\

El resultado ha sido una interfaz de usuario amigable para el usuario. A pesar de ello, existen mejoras que podrían aplicarse en un futuro al sistema desarrollado como el uso de un Docker para la instalación rápida y sencilla por cualquier usuario, el análisis del comportamiento de los animales con Deep Learning o el acceso de la interfaz de usuario desde cualquier lugar.