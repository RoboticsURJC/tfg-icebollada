\cleardoublepage

\chapter*{Resumen\markboth{Resumen}{Resumen}}
La robótica es un campo en auge que cada vez acompaña más a las personas en su día a día. Gracias al desarrollo tecnológico, esta disciplina está adquiriendo cada vez más importancia en diferentes campos en los que la robótica facilita y simplifica el trabajo de las personas debido a la eficiencia y automatización de sistemas que previamente eran manejados por humanos.\\

Para este Trabajo de Fin de Grado se ha creado un sistema de monitorización multisensorial para ratones de laboratorio bajo una plataforma de bajo coste. Se ha contactado con los investigadores del Laboratorio del Bienestar e Investigación Animal de la Universidad de Alcalá de Henares para el conocimiento de las carencias mejorables a través del presente trabajo. Como consecuencia se ha decidido crear una interfaz de usuario que ofrezca simplicidad a la hora de entender los datos y que reúna toda la información necesaria.\\

Para la obtención de este sistema, se ha utilizado la placa Raspberry Pi 4, que ha permitido que el sistema sea económicamente accesible. Distintos sensores se han utilizado para la medida de valores del entorno. Todos estos valores se han recogido bajo un script de Python que realiza lecturas paralelas de todos ellos. Por otro lado, se ha utilizado Flask para la creación de servidores web que permiten la visualización de las cámaras. Con el script de lecturas y los servidores web, se ha podido crear la interfaz de usuario en Node-Red.\\

Finalmente, se ha creado un dataset con 354 imágenes de ratones, que, junto a YOLO ha permitido entrenar un modelo de detección de ratones. Este modelo se ha entrenado a través de una red neuronal convolucional, un tipo de algoritmo de Deep Learning. Así pues se ha obtenido un sistema seguro que proporciona el control y visualización del entorno disponible desde cualquier dispositivo, dejando a su vez abiertas nuevas vías como el control del tiempo de las acciones de los animales mediante el uso del modelo de detección.