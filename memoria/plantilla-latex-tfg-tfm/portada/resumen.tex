\cleardoublepage

\chapter*{Resumen\markboth{Resumen}{Resumen}}

Desde hace siglos, la experimentación con animales se ha llevado a cabo para saber qué les sucede a los humanos y al mundo que les rodea. A día de hoy, aunque de una manera distinta debido tanto al avance de la humanidad como de la tecnología, esta experimentación sigue desarrollándose en diferentes ámbitos; para el estudio de los comportamientos de animales o el uso de éstos como modelo de investigación en diferentes campos, desde el testado de productos hasta la sanidad humana.\\

Los estudios con animales en la medicina han tenido gran importancia en el desarrollo de vacunas modernas como la tuberculosis o la meningitis ya que los animales, más concretamente los ratones, sufren enfermedades similares a los humanos. De esta manera, los ratones se han convertido en uno de los principales animales para estudiar su comportamiento y el efecto que éste puede tener en la detección de enfermedades neurológicas, como el autismo, el parkinson o el alzheimer.\\

La observación de estos animales en todo momento es una necesidad para analizar y estudiar su comportamiento, así como tener la información de las condiciones del entorno a las que se encuentran, haciendo necesario que haya trabajadores analizando las condiciones y grabaciones que tienen que hacerse a los ratones diariamente. Para evitar esto, el fin de este trabajo es crear un sistema dotando con los sensores necesarios al entorno de los roedores para obtener la información de forma automatizada y traducirlo en una interfaz comprensible para cualquier usuario, así como el control por vídeo a través de algoritmos de visión artificial para detectar los movimientos de los animales bajo una plataforma económicamente accesible para todo el mundo. \\

Durante todo el desarrollo del trabajo han surgido problemas en conexiones, seguridad, instalación, en el uso de algunos sensores y el uso de herramientas no usadas hasta el momento. Tras el estudio y la comprensión de los errores encontrados, así como con la ayuda de internet y de las experiencias de otros usuarios, tanto con problemas similares como con posibles soluciones, todos los errores han podido solucionarse.
