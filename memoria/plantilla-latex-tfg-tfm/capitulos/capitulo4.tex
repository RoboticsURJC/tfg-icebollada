\chapter{Diseño}
\label{cap:capitulo4}

\begin{flushright}
\begin{minipage}[]{10cm}
\emph{Quizás algún fragmento de libro inspirador...}\\
\end{minipage}\\

Autor, \textit{Título}\\
\end{flushright}

\vspace{1cm}

Escribe aquí un párrafo explicando brevemente lo que vas a contar en este capítulo. En este capítulo (y quizás alguno más) es donde, por fin, describes detalladamente qué has hecho y qué experimentos has llevado a cabo para validar tus desarrollos.

\section{Snippets}

Puede resultar interesante, para clarificar la descripción, mostrar fragmentos de código (o \textit{snippets}) ilustrativos. En el Código \ref{cod:codejemplo} vemos un ejemplo escrito en \texttt{C++}.

\begin{code}[h]
\begin{lstlisting}[language=C++]
void Memory::hypothesizeParallelograms () {
  for(it1 = this->controller->segmentMemory.begin(); it1++) {
    squareFound = false; it2 = it1; it2++;
    while ((it2 != this->controller->segmentMemory.end()) && (!squareFound)) {
      if (geometry::haveACommonVertex((*it1),(*it2),&square)) {
        dist1 = geometry::distanceBetweenPoints3D ((*it1).start, (*it1).end);
        dist2 = geometry::distanceBetweenPoints3D ((*it2).start, (*it2).end);
      }
    // [...]
\end{lstlisting}
\caption[Función para buscar elementos 3D en la imagen]{Función para buscar elementos 3D en la imagen}
\label{cod:codejemplo}
\end{code}

En el Código \ref{cod:codejemplo2} vemos un ejemplo escrito en \texttt{Python}.

\begin{code}[h]
\begin{lstlisting}[language=Python]
def mostrarValores():
    print (w1.get(), w2.get())

master = Tk()
w1 = Scale(master, from_=0, to=42)
w1.pack()
w2 = Scale(master, from_=0, to=200, orient=HORIZONTAL)
w2.pack()
Button(master, text='Show', command=mostrarValores).pack()

mainloop()
\end{lstlisting}
\caption[Cómo usar un Slider]{Cómo usar un Slider}
\label{cod:codejemplo2}
\end{code}

\section{Verbatim}

Para mencionar identificadores usados en el código ---como nombres de funciones o variables--- en el texto, usa el entorno literal o verbatim \verb|hypothesizeParallelograms()|. También se puede usar este entorno para varias líneas, como se ve a continuación:

\begin{verbatim}
void Memory::hypothesizeParallelograms () {
  // add your code here
}
\end{verbatim}

\section{Ecuaciones}

Si necesitas insertar alguna ecuación, puedes hacerlo. Al igual que las figuras, no te olvides de referenciarlas. A continuación se exponen algunas ecuaciones de ejemplo: Ecuación \ref{ec:ec1} y Ecuación \ref{ec:ec2}.

\begin{myequation}[h]
\begin{equation}
H = 1 - \frac{\sum_{i=0}^{N}\frac{(\frac{d_{j_s} + d_{j_e}}{2})}{N}}{M}
\nonumber
\label{ec:ec1}
\end{equation}
\caption[Ejemplo de ecuación con fracciones]{Ejemplo de ecuación con fracciones}
\end{myequation} 

\begin{myequation}[h]
\begin{equation}
v(entrada)= \left\{
	\begin{array}{lcc}
		0 & \mbox{if} & \epsilon_t < 0.1\\
		K_p\cdot{(T_{t}-T)} & \mbox{if}& 0.1 \leq \epsilon_t < M_t\\
		K_p \cdot M_t & \mbox{if}& M_t < \epsilon_t
	\end{array}
\right.
\label{ec:ec2}
\end{equation}
\caption[Ejemplo de ecuación con array y letras y símbolos especiales]{Ejemplo de ecuación con array y letras y símbolos especiales}
\end{myequation}

\section{Tablas o cuadros}

Si necesitas insertar una tabla, hazlo dígnamente usando las propias tablas de \LaTeX, no usando pantallazos e insertándolas como figuras... En el Cuadro \ref{cuadro:ejemplo} vemos un ejemplo.

\begin{table}[H]
\begin{center}
\begin{tabular}{|c|c|}
\hline
\textbf{Parámetros} & \textbf{Valores} \\
\hline
Tipo de sensor & Sony IMX219PQ[7] CMOS 8-Mpx \\
Tamaño del sensor & 3.674 x 2.760 mm (1/4" format) \\
Número de pixels & 3280 x 2464 (active pixels) \\
Tamaño de pixel & 1.12 x 1.12 um \\
Lente & f=3.04 mm, f/2.0 \\
Ángulo de visión & 62.2 x 48.8 degrees \\
Lente SLR equivalente & 29 mm \\
\hline
\end{tabular}
\caption{Parámetros intrínsecos de la cámara}
\label{cuadro:ejemplo}
\end{center}
\end{table}

