\chapter{Conclusiones}
\label{cap:capitulo5}

%\begin{flushright}
%\begin{minipage}[]{10cm}
%\emph{Quizás algún fragmento de libro inspirador...}\\
%\end{minipage}\\

%Autor, \textit{Título}\\
%\end{flushright}

\vspace{1cm}
Por último, en este capítulo se presentan las conclusiones que se han obtenido con el desarrollo del presente trabajo. Se recapitulan los objetivos marcados al inicio del proyecto, el proceso que se ha llevado para cumplirlos, así como los diferentes problemas que han surgido y las soluciones tomadas para abordarlos. Además, se presentan unas posibles líneas futuras de continuación de este trabajo.\\

\section{Conclusiones}
Con el desarrollo de este trabajo se han logrado los objetivos establecidos en el Capítulo \ref{cap:capitulo2}. El objetivo principal era obtener un sistema multisensorial que fuese capaz de monitorizar los ratones de laboratorio y que, además, fuese de bajo coste. La idea era que los investigadores del Laboratorio de Bienestar e Investigación Animal de la Universidad de Alcalá de Henares tuviesen un sistema que les aportase información de las condiciones (de bienestar) en la que se encuentran los animales, y que también les facilitase el trabajo de supervisión de estos por medio de un sistema de monitorización. Todo ello se ha conseguido de forma plausible.\\

También se han conseguido los tres objetivos específicos. En primero de ellos era recoger la lectura de los sensores de forma paralela en un mismo fichero. El segundo objetivo consistía en la creación de un servidor web por cada cámara para poder visualizarlas desde distintos dispositivos y, finalmente, el tercer objetivo era la detección de ratones mediante un algoritmo de reconocimiento de objetos.\\

Con la elección de la placa Raspberry Pi para el desarrollo de este trabajo, se ha conseguido que el sistema implementado sea económicamente accesible para cualquier usuario. Esta placa de bajo coste ha permitido la conexión de distintos sensores a través de los distintos puertos que posee, como pines GPIO y puertos USB.\\

La lectura de los sensores se ha volcado a un fichero, el cual recibe la lectura concurrente de todos ellos gracias al uso de la librería \verb|Threads|. Además, este fichero genera un fichero CSV con los datos recogidos en cada momento, por si es necesario disponer de un seguimiento a lo largo del tiempo. Con este fichero, además de cumplir el primer objetivo, se ha podido iniciar la creación de la IU. Con el uso de Node-Red, estas lecturas numéricas se han adaptado a \textit{widgets} comprensibles para el usuario final.\\

Con el uso de Flask se ha conseguido el segundo objetivo. Ha permitido crear dos servidores web independientes para la visualización de las imágenes registradas por la PiCam (Figura \ref{fig:picam_of}) y la cámara térmica (Figura \ref{fig:termicos}-b). Estos servidores han sido protegidos con 2FA, requiriendo un login a través de nombre, contraseña y número de autenticación generado a través de Google Authenticator. Se han podido integrar en la IU de Node-Red junto a los \textit{widgets} de las lecturas de los sensores. De esta forma, se ha obtenido la IU completa accesible desde cualquier dispositivo conectado a la misma red que la Raspberry Pi.\\

Para la consecución del último objetivo, se ha creado un dataset de ratones con 354 imágenes. Posteriormente, se ha entrenado un modelo con el uso de YOLOv5, que ha permitido la detección de ratones a través de la PiCam.\\

Este Trabajo Fin de Grado supone un avance importante respecto a los sistemas equivalentes que se pueden encontrar en el mercado, pues lo hace accesible económicamente a cualquier laboratorio de investigación animal. Se han cumplido todos los objetivos marcados, además de añadir otras características como seguridad o accesibilidad desde distintos dispositivos, que aportan comodidad y confianza al usuario que utiliza el sistema.\\

Durante la realización del trabajo, se han obtenido conocimientos sobre distintos aspectos que no se dominaban, como la creación de un servidor web, además de la seguridad necesaria que este requiere. Además, se han aplicado muchos de los conocimientos adquiridos durante el Grado, y han surgido problemas que han requerido de esfuerzo y comprensión del tema trabajado para su solución.\\

\section{Líneas futuras}
En esta sección se comentan algunas vías que pueden permitir la continuidad de este trabajo para ofrecer funcionalidades que actualmente no tiene:
\begin{itemize}
\item Adaptación de la IU a un servidor accesible desde cualquier dispositivo. Actualmente, la IU es accesible por cualquier dispositivo que esté conectado a la misma red. Una funcionalidad útil sería poder acceder a ella desde cualquier sitio, para poder controlar el sistema en cualquier momento y desde cualquier lugar.
\item Análisis del comportamiento de los animales en las cubetas mediante técnicas de Deep Learning. Por ejemplo, realizar un control automático del tiempo que pasa cada ratón en los distintos sitios de la cubeta, o bien en qué cubeta pasan más tiempo los ratones permitirían a los investigadores poder realizar un análisis del comportamiento de los animales a través de la detección de ratones creada, y permitiendo identificar a los ratones individualmente, se podría incorporar esta idea al sistema desarrollado.
\item Creación de un Docker, para permitir la instalación del sistema a cualquier usuario. Con un Docker que instale todas las librerías y active los requisitos necesarios para el funcionamiento del sistema, cualquier usuario podría instalar y utilizar este sistema de una forma rápida y sencilla.
\end{itemize}

%\section{Corrector ortográfico}

%Una vez tengas todo, no olvides pasar el corrector ortográfico de \LaTeX a todos tus ficheros \textit{.tex}. En \texttt{Windows}, el propio editor \texttt{TeXworks} incluye el corrector. En \texttt{Linux}, usa \texttt{aspell} ejecutando el siguiente comando en tu terminal:

%\begin{verbatim}
%aspell --lang=es --mode=tex check capitulo1.tex
%\end{verbatim}
